% Enable warnings about problematic code
\RequirePackage[l2tabu, orthodox]{nag}

\documentclass{WeSTassignment}

% The lecture title, e.g. "Web Information Retrieval".
\lecture{Introduction to Web Science}
% The names of the lecturer and the instructor(s)
\author{%
  Prof. Dr.~Steffen~Staab\\{\normalsize\mailto{staab@uni-koblenz.de}} \and
  Ren{\'e}~Pickhardt\\{\normalsize\mailto{rpickhardt@uni-koblenz.de}} \and
   Korok~Sengupta\\{\normalsize\mailto{koroksengupta@uni-koblenz.de}} \and 
   Olga~Zagovora\\{\normalsize\mailto{zagovora@uni-koblenz.de}}
}
% Assignment number.
\assignmentnumber{12}
% Institute of lecture.
\institute{%
  Institute of Web Science and Technologies\\%
  Department of Computer Science\\%
  University of Koblenz-Landau%
}
% Date until students should submit their solutions.
\datesubmission{February 15, 2016, 10:00 a.m.}
% Date on which the assignments will be discussed in the tutorial.
\datetutorial{February 17, 2016, 12:00 p.m.}

% Set language of text.
\setdefaultlanguage[
  variant = american, % Use American instead of British English.
]{english}

% Specify bib file location.
\addbibresource{bibliography.bib}

% For left aligned centered boxes
% see http://tex.stackexchange.com/a/25591/75225
\usepackage{varwidth}

% ==============================================================================
% Document

\begin{document}

\maketitle
This assignment is about \textbf{Net Neutrality \& Copyright}

Copying answers straight way from any source wont be considered for the final score of this assignment. Please cite your sources if any.\\ \\ 

%Please mention your team Names here: 
Team Name: hotel

Andrea Mildes - mildes@uni-koblenz.de

Sebastian Blei - sblei@uni-koblenz.de

Johannes Kirchner - jkirchner@uni-koblenz.de

Abdul Afghan - abdul.afghan@outlook.de


\section{Copyright \& Creative Commons (10 points)}

\subsection{Differences}
On what grounds can you differentiate between Copyright and Creative Commons?

\subsubsection{Case Study}
Let us consider that Donald has an idea to develop a study material for the poor  children from his area who cannot attend a school. But in order to have this idea as a product, he needs some financial help from investors so that he can collect materials and also to set up a website where kids can study for free using the materials and videos he makes for them. But Donald wants it to be completely free and shareable so that it can help anyone. 

\begin{itemize}
\item Can Donald's \emph{idea} be copyrighted?
\item How can Donald protect his idea when he presents it to the investors?
\item Since the investors are investing capital, can they still recover money if Donald wants to go for the Creative Common licenses? If so, state the ways?  
\end{itemize}
\emph{Solution:}\\
\emph{1.1 Differences between copyright and creative commons}
When something has a copyright, it gives a lot of power to the license owner. He / She can decide what is allowed and what not whereas the consumer has no power at all. The consumer can license a product, however he doesn't own it. In some cases the license owner actually want's it that way, but sometime the owner may want to allow other people to use his product, but doesn't want to negotiate a license agreement with every single person interested in using his product. In this case, the license owner can provide his copyrighted product under the creative common license. So CC doesn't contradict copyright, it extends it and grants more rights to the user. 

\emph{1.1.1 Case Study}\\
\begin{itemize}
	\item \emph{Can Donald's idea be copyrighted?}: No, ideas can't be copyrighted.
	\item \emph{How can Donald protect his idea?}: Donald could patent his idea in order to protect it. However since patents are usually very expensive, this might not be an option for Donald, which makes things more difficult. In case of doubt, Donald could sign an non disclosure agreement with everyone he presents his idea to, wich could protect his idea in the short run.
	\item \emph{Can investors recover money?}: Yes, there are many ways to monetize work that is copyrighted under CC. For example, you could use the creative commons non commercial license. In this case, your work can only used for non commercial purposes. So you could still sell your work in case the customer want's to your work for an commercial purpose. Moreover you could only make some part of your work available for free and charge for additional parts. For example, all digital version of Donalds materials could be available for free, but he could still sell Books containing this materials. Furthermore, Donald could decide to rely on donations and make money this way. 
\end{itemize}

%-------------------------------------------------------------------------------

\section{Neutrality(10 points)}

\begin{itemize}
\item Define the term \emph{net neutrality}.
\item Argue for and against the motion on the concept of priority pricing as discussed in Kögler et.al(2011)\footnote{\label{note1}Berger-Kögler, U. and Kruse, J. (2011) ‘Net neutrality regulation of the internet?’, Int. J. Management and Network Economics, Vol. 2, No. 1, pp.3–23.}
\item - Explain why?\quote{"...additional internet capacity would not lead to additional revenues because of the flat rates."}\textsuperscript{\ref{note1}}

\end{itemize}


\emph{Solution:}\\
\emph{2.1 Define net neutrality} \\
Net neutrality is achieved when an ISP merely acts as a service provider and delivers data to its destination without interfering or knowing what it is delivering. Every package is treated equally at every router without regard to the sender or receiver. Not net neutral would be an ISP that looks at every package and prioritizes traffic depending on which sender sent the corresponding package. \\
\emph{Argue for and against priority pricing:}\\
Implementing priority pricing concepts could come at a high cost. First, some service providers will be discriminated, because they can't afford paying ISP's for prioritizing their traffic. This would in return favor the "big players" and strengthen their market position. This is also an offense against the free internet because it would result in a "two-class internet" where only the rich can afford an acceptable user experience while the poor can't. This will also harm charitable organizations, since money is usually rare in that segment. Furthermore, internet service providers should not interfere with the content flow since they should be (in our opinion) mere infrastructure providers and should act like "gate-keepers" and decide wich traffic reaches a user in what speed. \\ However, there are - as always - some arguments in favor of priority pricing. First of all, it counteracts the problem of overloading the internet, since some services are slowed down artificially. Also, the earned money could be reinvested in a better internet infrastructure which is good for every user. Finally in most cases, the user won't even notice the difference in internet speed because he / she uses only popular (and paying) services anyways.



\makefooter

\end{document}
